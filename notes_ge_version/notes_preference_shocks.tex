\documentclass[pdftex,11pt]{article}
\usepackage[pdftex]{graphicx,color}
\usepackage{setspace,multirow}
\usepackage{amsmath,amssymb}
\usepackage{titlesec}
\usepackage{subfigure}
\usepackage{fancyhdr}
\usepackage[longnamesfirst]{natbib}
\usepackage{cite}
\usepackage[paperwidth=8.5in,
left=0.5in,right=0.5in,paperheight=11.0in,textheight=9.5in,centering]{geometry}
\usepackage{booktabs}
\usepackage{rotating}

\bibliographystyle{ecta}
\definecolor{nblue}{RGB}{0,0,128}

\usepackage[colorlinks=true, linkcolor=nblue,
citecolor=black, urlcolor=nblue, bookmarks=false,
pdfstartview={XYZ null null 0.90},
pdftitle={Redistributing the Gains From Trade Through Progressive Taxation},
pdfauthor={Spencer G. Lyon, Michael E. Waugh},
pdfkeywords={economics, trade, dynamics, quant econ, lyon, waugh, incomplete markets, taxes, redistribution, progressivity, inequality, Ricardo, julia, migration, taxation, social insurance}
]{hyperref}

\usepackage{setspace}

\onehalfspace

\renewcommand{\baselinestretch}{1.1}
\renewcommand{\arraystretch}{.7}
\setlength{\parindent}{0em}
\setlength{\parskip}{.1in}
\renewcommand\familydefault{\sfdefault}

\titleformat{\section}{\large\bf}{\thesection.}{.5em}{}
\titleformat{\subsection}{\normalsize\bf}{\thesubsection.}{.5em}{}
\titleformat{\subsubsection}{\normalsize\bf}{\thesubsubsection.}{.5em}{}

\def\thesection{\arabic{section}}
\def\thesubsection{\Alph{subsection}}
\def\thesubsubsection{\Roman{subsubsection}}

\newtheorem{proposition}{Proposition}
\newtheorem{assumption}{Assumption}

\pagestyle{fancy}
\rhead{}
\lhead{}
\cfoot{\thepage}
\lfoot{}
\lfoot{Revised: \today}
\renewcommand{\headrulewidth}{0pt}


\begin{document}

\subsection{Labor Earnings in the Rural Area}

The model is \textbf{solved} in the following way. This discussion below is simply about code, not to be in the paper. Labor earnings in the rural area are
\begin{align}
w_{r}(s, i, N_r) = A_{i} s
\end{align}
where $A_i$ is mechanically the seasonal productivity shock. But it is a convolution of the wage per efficiency unit of labor AND the seasonal shock. Then given the production function we can invert what the actual wage per efficiency unit is given the number of efficiency units that are working in the rural area. The number of efficiency units working in a rural area in season $i$ is:
\begin{align}
N_{r,i} = \int_{r,i} s \ ds
\end{align}
so this is simply adding up the efficiency units of those working in the rural area. Then since wages equal the marginal product of labor we have:
\begin{align}
w_{r}(s, i, N_r) = \alpha \tilde{A}_{i} \left( N_{r,i} \right)^{\alpha - 1} s \label{eq:wage_mpl}
\end{align}
and then given an $\alpha$ we can invert (\ref{eq:wage_mpl}) to find the ``fundamental'' seasonal productivity, $\tilde{A}_{i}$, i.e. given $A_i$ and equilibrium $N_{r,i}$, we can figure out what $\tilde{A}_{i}$. This is important as it allows us to feed in the value $\alpha \tilde{A}_{i} \left( N_{r,i} \right)^{\alpha - 1}$ into the code and the resulting output will exactly replicated our calibrated economy.

\textbf{Calibrating the wage elasticity.} This is how I did it\ldots take the control village with $N_{r,m}$ efficiency units of labor; take $N^{ex}_{r,m}$ efficiency units of labor from the experiment. Then compute the log difference in wages using (\ref{eq:wage_mpl}):
\begin{align}
\log w^{ex}_{r}(s, m, N_r) - \log w_{r}(s, m, N_r) \label{eq:wage_log_diff}
\end{align}
and given that we know the change in \textbf{bodies}, we can compute a wage elasticity. In particular, we phrase it interms of the change in migration flows as in ACM. So for every 10 percent increase in migration, we should see a 2.2 percent increase in wages. In our calibrated economy, there is a 22 percent increase in migration rates between the experiment and control, so we want to find an $\alpha$ such that \ref{eq:wage_log_diff} takes on the value of 4.4 percent.

\subsection{GE Counterfactual}

This is the way we will implement the counterfactual. First, in the calibrated economy, we will record the level of assets that satisfies the ACM sampling criteria, call this $a^{acm}$. Recall that this is found so that only the bottom 50th percentile of the rural asset distribution are offered the travel subsidy. In the GE counterfactual, we will provide a permanent travel subsidy to those who fall below $a^{acm}$. This is done for several reasons, but most important it makes sure the we are looking at ``similar guys'' as in the PE experiment. The other thing is that in our model there is a strong tendency if the cost goes away for many people to relocate back to the rural area; this will modulate this. Second, given the change in costs, we find wages that which are consistent with \ref{eq:wage_mpl}.

So note that there are two important aspects of this policy. First, is the permanence of it. This will change extensive margin movements, i.e. location choice, migration choice, and the accumulation of experience. Second, through these changes it will affect wages. For example, if seasonal migration rates increase, then this will benefit those who stay as wages will increase. Note that this can also cut the other way as if more people decide to locate in the rural area, non-monga wage rates will \emph{decrease} hurting most households.

\subsection{GE Counterfactual Results}

Table \ref{ta:welfare_quintile} compares the baseline welfare gains from the one-time experiment to the implementation of the policy. The middle two columns reports the welfare gains when rural wages are fixed, the final two columns report when wages adjust. 

Several features stand out. First, general equilibrium affects work against the welfare gains, but by relatively small margins. This can be seen by comparing the results across the columns with fixed prices vs. the final two columns with changes in prices. GE affects are small and work against the welfare gains for several reasons. First, because seasonal migration does increase by a large amount, the production technology implies that wages in the rural area increase benefiting those who do not move. This is true with wages increasing by about three percent in the monga season relative to the baseline economy. However, this force works also in the opposite direction during the good season. Because of the change in policy, more people decide to relocate and live in the rural area. As Table \ref{ta:welfare_quintile} shows, the rural population actually increases to around 66 percent. This extensive margin movement in the population decreases wages for those in the rural area in the good season by about one percentage point. Moreover, this affect has a larger impact on the rural population because it is affecting a larger base since few people move out in the good season.

\begin{table}[h]
\setlength {\tabcolsep}{1.45mm}
\renewcommand{\arraystretch}{1.75}
\begin{center}
\caption{\textbf{Consumption-Equivalent Welfare Gains by Income Quintile }\label{ta:welfare_quintile}}
\begin{tabular}{c c c c c c c c c c c}
\hline
\hline
& & \multicolumn{2}{c}{Conditional Migration Transfer} && \multicolumn{2}{c}{Permanent Transfer, Fixed Prices} && \multicolumn{2}{c}{Permanent Transfer, GE}\\
\cmidrule(lr){3-4} \cmidrule(lr){5-7}     \cmidrule(lr){8-10}    
& & \small Welfare  &\small Migr. Rate  && \small  Welfare  &\small Migr. Rate && \small  Welfare  &\small Migr. Rate \\
\multirow{5}{*}{\rotatebox{90}{\small Income Quintile}} & 1 & 1.01  & 85.8 &&4.80&92.6&&4.41& 91.1 \\
                                                        & 2 & 0.35  & 59.1 &&3.79&78.7&&3.43& 76.2 \\
                                                        & 3 & 0.21  & 48.8 &&3.46&69.7&&3.09& 66.5 \\
                                                        & 4 & 0.13  & 40.9 &&3.15&60.8&&2.81& 57.8 \\
                                                        & 5 & 0.07  & 35.8 &&2.69&50.6&&2.38& 47.8 \\
\hline
\multicolumn{2}{l}{\small Average} &0.35  &56.0 && 3.58 &70.0 &&  3.20 &  67.9\\
\multicolumn{2}{l}{\small Fraction Rural} &  &60.0 &&  &66.0&&   &  65.3\\
\multicolumn{2}{l}{\small Fraction w. Experince} &  &23.0 &&  &42.0&&   &  65.3\\

\hline
\end{tabular}
\parbox[c]{6.0in}{%
{\footnotesize  \vspace{0.25cm} Note: The table reports the (lifetime) consumption-equivalent welfare gains from the conditional migration transfers relative to an unconditional transfer program costing the same total amount. The numbers in the table are the average percent increase in consumption each period that would make the households indifferent between the consumption increase and the transfers.}
}
\end{center}
\end{table}


With that said, the welfare gains from this policy are large. We suspect that there are several reasons why they are so large. First, is the permanence of it and, in particular, the ``insurance'' that subsidized migration provides. This can be seen by noting that for those at the top income quintile, the one time transfer delivers a very tiny benefit, less than a tenth of a percentage point. But the permanent transfer delivers a huge gain, more than two percent. Because these guys have some assets and relatively high income, the transfer is benefiting them not by allowing them to migrate today, but allowing them to migrate more easily in the future in states of the world where it is valuable.

A second issue as to why the welfare gain is large is the endogenous change in experience. The permanent migration transfer increases the number of households who have experience with the urban area. As Table \ref{ta:welfare_quintile} shows, the fraction of the rural population with experience essentially doubles. This is because, in the new stationary distribution, a lower monetary cost allows more households to reduce they utility cost as they migrate and acquire more experience. 

This dynamic complimentarily between experience and the transfer does two things. First, migration is now less costly for most households as they enjoy the transfer and the disutility of the urban area no longer hurts. This dynamic complimentarily between experience and the transfer can be seen in the fact the migration rates are more than ten percentage points larger than then that associated with the one time transfer. This dynamic complimentarily also implies that when households do migrate to the urban area,




\newpage

Before describing the value functions of a household, it is important to have a complete accounting of the state space. The state variables for a household can be divided into objects that are permanent, transitory, endogenous and aggregate.
\begin{itemize}
\item \textbf{Permanent productivity state.} Each household is endowed with $z$ efficiency units in the urban area and one efficiency unit in the rural area. This is the ``static Roy model'' aspect of the model.

\item \textbf{Transitory productivity state.} Each household is subject to transitory productivity shocks, $s$.

\item \textbf{Transitory moving shock.} Each household is subject to i.i.d. moving shocks $\nu$.

\item \textbf{Endogenous state variables.} There are three endogenous (individual) state variables. The first is the household's asset holdings, $a$. The second is a composite variable that describes the household's location and migration status. The possible states are: rural, seasonal-migrant (living in the rural area but working in the urban area for one period), and urban. The third is whether or not the household is an inexperienced migrant, $x$, and, thus, whether or not it suffers disutility $\bar u$ from locating in the urban area.

\item \textbf{Aggregate state variables.} There are two aggregate state variables: the season, $i \in \{g,\ell \}$, and the number of workers in the rural area, $N_r$. The season determines the current and future productivity in the rural area, and jointly, the two aggregate states determine the current wage per efficiency unit as in equation \eqref{eq:wage_per_efficiency_units}.
\end{itemize}
We begin with the problem of a rural household. Because $z$ is time-invariant for each household, we omit it from the formulation of the household's problem below.

\textbf{Rural Households.} A rural household with productivity $z$ solves the following problem:
\begin{align}
\small
v(a, r, s, \nu, x, i, N_r) =& \nonumber \\
  \max \bigg\{ \ v(a, r,  s, & x, i, N_r| \ \mbox{stay}) + \nu^{\tiny\mbox{stay}},  \ v(a, r, s, x, i, N_r | \ \mbox{seas}) + \nu^{\tiny\mbox{seas}},\  v(a, r, s, x, i, N_r | \ \mbox{perm})+ \nu^{\tiny\mbox{perm}} \ \bigg \},
\label{eq:value_fun_rural}
\end{align}
where a household chooses among staying in the rural area, seasonally moving, and permanently moving. Influencing this choice is the value function associated with each option and the household's taste shock associated with each choice. Here we will follow the literature and assume that these taste shocks are independently and identically distributed across time and is distributed Type 1 extreme value distribution with scale parameter $\sigma_{\nu}$.

The distributional assumption on the taste shocks implies that the choice probability to stay in a location, for example, is:
\begin{eqnarray*}
\small
P(a, r, s, x, i, N_r| \ \mbox{stay}) = \frac{\exp\{\sigma_{\nu}^{-1} v(a, r,  s, x, i, N_r | \ \mbox{stay}\}}{\sum_{j_r} \exp\{\sigma_{\nu}^{-1} v(a, r,  s, x, i, N_r, | \ j_r)\}}
%\frac{\exp\{ \sigma_{\nu}^{-1} v(a, r,  s, & x, i, N_r, | \ \mbox{stay})\}}{\exp\{ \sigma_{\nu}^{-1} v(a, r,  s, & x, i, N_r, | \ \mbox{stay})\}
\end{eqnarray*}
where the sum across $j$'s are the different choices. Here the scale parameter shows up and modulates the strength of the preference shock in determining the move. For example, if $\sigma_{\nu}$ goes to infinity, then only the shock matters for the moving choice, and the probability of each individual choice is simply one over the number of choices. Another important feature to note with this specification is that the expected value function (with respect to the preference shocks) is
\begin{eqnarray*}
\small
E_{\nu} v(a, r, s, \nu, x, i, N_r) = \sigma_{\nu} \log \left( \sum_{j_r} \exp\{\sigma_{\nu}^{-1} v(a, r,  s, x, i, N_r, | \ j_r)\} \right).
\end{eqnarray*}
An important feature here is that this is not a simple probability weighted sum. The reason is that the taste shock is ``felt'' and the household is choosing over it. In other words, it's the expectation over the max operator, hence, this funky feature.

Below, the value functions conditional on a choice are described.

Conditional on staying in the rural area, the value function is:
\begin{align}
v(a, r, s, x, i, N_r | \ \mbox{stay}) =  \max_{a'\in \mathcal{A}}\bigg  \{ u(Ra + w_{r}(s, i, N_r) - a' )  + \beta \, \mathbb{E} [v(a',r, s',\nu', x',i',N_r')]  \bigg\},
\label{eq:bellman_rural_stay}
\end{align}
which says that the household chooses future asset holdings to maximize the expected present discounted value of utility. The asset holdings must respect the borrowing constraint and, thus, must lie in the set $\mathcal{A}$. Given asset choices, a household's consumption equals the gross return on current asset holdings, $Ra$, plus labor income from working in the rural area, $w_{r}(z, s, i)$, minus future asset holdings. Next period's state variables are the new asset holdings, location in the rural area, the transitory productivity shock, the experience level, the subsequent season, and the aggregate rural efficiency units in the next period. The expectation operator is defined over two uncertain outcomes: the transitory shocks and the change in experience. Recall, that if the household is experienced, it stays that way with probability $\pi$ and becomes inexperienced with probability $1-\pi$; if the household is inexperienced, then it stays inexperienced.

The value function associated with a permanent move is:
\begin{align}
v(a, r, s, x, i, N_r | \ \mbox{perm})  = \max_{a'\in \mathcal{A}} \bigg\{ u(Ra + w_r(z, s, i, N_r) - a' - m_{p} )  + \beta \, \mathbb{E} [v(a',u, s',\nu', x', i',N_r')] \bigg\}.
%\label{eq:bellman_rural_perm_move}
\nonumber
\end{align}
While similar to the staying value function, there are several points of difference. First, the agent must pay $m_p$ to make the permanent move, and this costs resources. Second, the continuation value function denotes that the household's location changes from the rural to the urban area.

\textbf{NEED TO THINK ABOUT!!!} What to do with the shock when the guy is on a seasonal move. Does he get hit with a preference shock? Does it make a difference as I think with one choice, the expected value just becomes the regular value function...

The value function associated with a seasonal move is:
\begin{align}
\hspace{-0.60cm}v(a, r, s, x, i, N_r | \ \mbox{seas}) = \max_{a'\in \mathcal{A}}&\bigg\{ u(Ra + w_r(s, i, N_r) - a' - m_T ) + \beta \, \mathbb{E} [v(a',seas, s', x', i', N_r')] \bigg\}.
\label{eq:bellman_rural_seas_move}
\end{align}
If a household decides to move seasonally, it pays the moving cost $m_T$, and works in the urban area in the next period. The key distinction between the permanent move and the seasonal move is that the seasonal move is for just one period. Hence, the location state variable is $seas$ and not $u$, as this indicates that the household is going to work in the urban area and return in the next period. The value function associated with a seasonal move while in the urban area is:
\begin{align}
v(a',seas, s',x', i',N_r')&=  \max_{a''\in \mathcal{A}}\bigg[ u(Ra' + w_u(z, s') - a'')\bar u^{x'} + \beta \, \mathbb{E} [v(a'',r, s'', \nu'', x'', i'',N_r'')] \bigg].
\label{eq:bellman_rural_sm}
\end{align}
There are several important points to take note of in (\ref{eq:bellman_rural_sm}). First, this household has only one choice: how to adjust its asset holdings. By the definition of a seasonal move, the household works in the urban area for one period and then returns to the rural area. Second, note how the disutility from living in the urban area appears (i.e., the presence of $\bar u$). Moreover, the state variable of a household's experience $x$ determines whether or not the disutility is experienced.

Equations (\ref{eq:bellman_rural_seas_move}) and (\ref{eq:bellman_rural_sm}) illustrate the forces that shape the decision to move seasonally and, in turn, our inferences from the experimental and survey results. Generally, the choice to move seasonally will relate to a household's comparative earnings advantage in the urban area relative to the rural area. However, several forces may lead a household with a permanent comparative advantage in the city not to move. First, the urban disutility may prevent the household from moving, even though its comparative advantage in the urban area is expected to be high. Second, there is risk associated with the move. A household does not know $s'$, and, hence, there is a chance that the income realization in the urban area will not be favorable. Third, the household may have limited assets that simply make a move infeasible or not sufficient to insure against a bad outcome in the urban area.

\textbf{Urban Households.} Urban households face problems similar to those described above, though they choose between just two options: staying or making a permanent move. For a household with productivity level $z$, the problem is:
\begin{align}
v(a, u, s, \nu, x, N_r, i) = \max \bigg\{ \ v(a, u, s, x, N_r, i| \ \mbox{stay}) + \nu^{\tiny\mbox{stay}}, \  v(a, u, s, x, N_r,  i| \ \mbox{perm}) + \nu^{\tiny\mbox{perm}} \ \bigg \}.
\label{eq:bellman_urban}
\end{align}
Again, influencing this choice is the value function associated with each option and the household's taste shock associated with each choice. These taste shocks are independently and identically distributed across time and distributed Type 1 extreme value distribution with the same scale parameter $\sigma_{\nu}$. The distributional assumption on the taste shocks implies that the choice probability to stay in a location, for example, is:
\begin{eqnarray*}
\small
P(a, u, s, x, i, N_r| \ \mbox{stay}) = \frac{\exp\{\sigma_{\nu}^{-1} v(a, u, s, x, N_r, i| \ \mbox{stay})  \}}{\sum_{j_u} \exp\{\sigma_{\nu}^{-1} v(a, u, s, x, N_r, i| \ j_u)\}}
%\frac{\exp\{ \sigma_{\nu}^{-1} v(a, r,  s, & x, i, N_r, | \ \mbox{stay})\}}{\exp\{ \sigma_{\nu}^{-1} v(a, r,  s, & x, i, N_r, | \ \mbox{stay})\}
\end{eqnarray*}
where the sum across $j_u$'s are the different choices available to the urban household. As above the expected value function (with respect to the preference shocks) is
\begin{eqnarray*}
\small
E_{\nu} v(a, u, s, \nu, x, i, N_r) = \sigma_{\nu} \log \left( \sum_{j_u} \exp\{\sigma_{\nu}^{-1} v(a, u,  s, x, i, N_r, | \ j_u)\} \right).
\end{eqnarray*}
\textbf{NOTE} Mechanically, this is lowering utility in the urban area because they have less options. In the urban area you get two draws, yet in the rural area you get three. From a preference standpoint, the rural area will naturally deliver higher expected utility, and will be a force to keep people in the urban area. This force has to be present in the old version of the model, i.e. the rural area will deliver more option value, but this will amplify the option value.




Conditional on staying in the urban area, the value is:
\begin{align}
\hspace{-0.60cm}v(a, u, s, x, i, N_r, | \ \mbox{stay}) =  \max_{a'\in \mathcal{A}}\bigg\{ u(Ra + w_{u}(z,s) - a' )\bar u^{x} + \beta \, \mathbb{E} [v(a',u,s',x',i',N_r')] \bigg\}.
\label{eq:bellman_urban_stay}
\end{align}
Households staying in the urban area have several key differences from those staying in the rural area. First, their wage depends on their permanent productivity level, $z$, and not on the season or number of aggregate efficiency units in the rural areas. Moreover, the transitory productivity shocks may have more or less volatility relative to the rural area, as modulated by the $\gamma$ parameter (see equation (\ref{eq:wages})). Third, the disutility from living in the urban area appears (i.e., the presence of $\bar u$), and the state variable of a household's experience $x$ determines whether or not the disutility is experienced.

Finally, as with rural households, expectations are taken with respect to the transitory shock $s$ and the change in experience. However, as these households are in the urban area, inexperienced households stay that way in the next period with probability $\lambda$ and become experienced with probability $1-\lambda$. Experienced households retain their experience.

The value function associated with a permanent move back to the rural area is:
\begin{align}
\hspace{-0.60cm} v(a, u, s, x, i, N_r | \ \mbox{perm}) = \max_{a'\in \mathcal{A}} \bigg[ u(Ra + w_{u}(z,s) - a' - m_{p} )\bar u^{x} + \beta \, \mathbb{E} [v(a',r, s', x', i',N_r')] \bigg].
\label{eq:bellman_urban_perm_move}
\end{align}
Here, the agent must pay $m_p$ to make the permanent move. Furthermore, the continuation value function denotes the household's location changes from the urban to the rural area. After a permanent move to the rural area, experienced households keep their experience with probability $\pi$ and lose it with probability $1-\pi$.

\end{document}  